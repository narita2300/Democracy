\subsection{Data}

\textcolor{red}{Note: Copy and pasted the entire Data section from Frey's paper, Acemoglu's data section, Hall and Jones' data section, and the Voice of the Future paper. }

Our analysis requires four types of data. First, we need data on Covid-19-related outcomes to measure performance. Second, we need data on democracies. Third, we need data on characteristics of countries. We use these pieces of information to find the association between Covid-19-outcomes and democratic regimes. Finally, we need IV data to determine the causal effect of democracy. Below, we describe the data sets we use. Table \ref{tab:descriptive-stats} provides descriptive statistics. 

\textbf{Covid-19-related Outcomes Data:} We use two indicators to measure Covid-19 performance. The first is Covid-19 mortality. The Covid-19 Data Repository Center for Systems Science and Engineering (CSSE) at Johns Hopkins University provides daily-updated data on confirmed cases and deaths for all countries. It collects data from administrative sources \citep{covid-data}. We use this data to obtain the total number of confirmed deaths attributed to Covid-19 in 2020.The second indicator is the GDP growth rate in 2020. The World Economic Outlook report by the \citet{imf} has data on the annual GDP growth rates (including predictions) valued at purchasing power parity from 1980 to 2025. Since the official GDP growth rates for 2020 is not publicly available yet, we use the predicted rate. 

\textbf{Democracy Data:} We adopt three measures of democracy. The first is the democracy index by the \citet{eiu}. It is constructed from five interrelated categories: electoral processes and pluralism, functioning of government, political participation, political culture, and civil liberties. The second measure is the representative government index from the Global State of Democracy 2019 by \citet{gsd}. It measures the extent to which popular elections for legislative and executive offices are contested and inclusive. The third measure is the polity index from Polity IV 2018 by the \citet{polity}. It is constructed by subtracting the autocracy score, which measures institutionalized autocracy, from the democracy score, which measures institutionalized democracy. We normalize all indices by their standard deviations. 

\textbf{Country Characteristics Data:} To control for country characteristics in our regressions, we collect data for absolute latitude, mean temperature, mean precipitation, GDP per capita and population density. For further details refer to Table \ref{tab:sources} in the Appendix. 

\textbf{IV Data:} To identify the causal relationship between democracy, we adopt IVs used by \citet{ajr} and \citet{hj}. European settler mortality is the mortality rate of soldiers, bishops, and sailors stationed in the colonies between the seventeen and nineteenth centuries. It is measured in terms of deaths per annum per 1,000 mean strength. \footnote{Raw mortality numbers are adjusted to what they would be if a force of 1,000 living people were kept in place for a whole year. It is possible for this number to exceed 1,000 in episodes of extreme mortality as those who die are replaced with new arrivals.} The fraction speaking English is the fraction of a country's population speaking English as a mother tongue, and the fraction speaking European is the fraction speaking one of the five primary Western European languages (English, French, German, Portuguese, and Spanish) as a mother tongue. The Frankel-Romer trade share is the predicted trade share of an economy, based on the gravity model of international trade. Further discussions of these IVs are in Section \ref{sec:causal}.

\begin{comment}
{\color{blue}
We build a dataset allowing us to trace the daily spread of Covid-19 cases, government’s re- sponse to the pandemic, and the movement of people across 111 countries over the entire lockdown period to date. Data on movement and travel were collected from Google’s Community Mobility Reports, and matched with information on policy restrictions, testing, and tracing from the Oxford Covid-19 Government Response Tracker (OxCGRT) (Hale et al., 2020). Table 1 provides some summary statistics for the variables of interest in our analysis.}

We build a dataset to compare total Covid-19 deaths and GDP growth from 2019 to 2020 across 156 countries. Data on Covid-19 deaths were collected from Data Repository Center for Systems Science and Engineering (CSSE) at Johns Hopkins University and GDP growth rates were obtained from the \citet{imf}. Table \ref{tab:descriptive-stats} provides descriptive statistics for the key variables of interest. 

{\color{blue}
The Google Community Mobility Reports provide daily data on Google Maps users who have opted-in to the ”location history” in their Google accounts settings across 132 countries. The reports calculate changes in movement compared to a baseline, which is the median value for the corresponding day of the week during the period between the 3rd of January and the 6th of February 2020. The purpose of travel has been assigned to one of the following cate- gories: retail and recreation, groceries and pharmacies, parks, transit stations, workplaces, and residential.

OxCGRT is a novel dataset which is published by the Blavatnik School of Government at the University of Oxford. It contains various lockdown measures, such as school and work- place closings, travel restrictions, bans on public gatherings, and stay-at-home requirements, etc. These measures are complied into a stringency index, which is constantly updated to re- flect daily changes in policy. This allows us to analyse policy changes as well as geographic mobility patterns on a daily basis. Data on testing policy and contact tracing is also taken from OxCGRT.}

The Covid-19 Data Repository Center for Systems Science and Engineering (CSSE) at Johns Hopkins University provides daily-updated data on confirmed cases and deaths for all countries, which are sourced from administrative sources \citep{covid-data}. We use this data to obtain the total number of confirmed deaths attributed to Covid-19 in 2020. It is described in the first row of Table \ref{tab:data-sources}.

The World Economic Outlook report by the \citet{imf} contains data on the annual GDP growth rates (including predictions) valued at purchasing power parity from 1980 to 2025. Since the official GDP growth rate for 2020 is not publicly available yet, we use the GDP growth rate from 2019 to 2020. It is described in Table \ref{tab:descriptive-stats}'s second row. 

{\color{blue}
To measure democratic institutions, we collect data from two sources. Following BenY-ishay and Betancourt (2014), who argue that democracy constitutes both civil and political rights, we use the civil and political rights country score from Freedom in the World 2020, compiled by Freedom House. The second variable is a dummy variable equal to 1 if a country classified as authoritarian, taken from Dictatorship Countries Population 2020, compiled by the World Population Review.}

The third to fifth rows describe the three measures of democratic institutions. The first variable is the democracy index compiled by the \citet{eiu}. It is constructed from five interrelated categories: electoral processes and pluralism, functioning of government, political participation, political culture, and civil liberties.The second variable is the representative government index from the Global State of Democracy 2019 compiled by \citet{gsd}. It measures the extent that popular elections for legislative and executive offices are contested and inclusive. The third variable is the polity index from Polity IV 2018 by the \citet{polity}. It is constructed by subtracting the autocracy score, which measures institutionalized autocracy, from the democracy score, which measures institutionalized democracy.

The next five rows give descriptive statistics for the control variables: absolute latitude, mean temperature, mean precipitation, GDP per capita and population density. For details on data sources, refer to Table \ref{tab:sources}. 

The last four rows describe the IVs. Log European settler mortality is sourced from \citet{ajr}. The remaining IVs are sourced from \citet{hj}. Further details of these IVs are discussed in Section \ref{sec:causal}. 

\end{comment}


\begin{comment}

{\color{blue} Acemoglu: Table 1 provides descriptive statistics for the key variables of interest. The first column is for the whole world, and column (2) is for our base sample, limited to the 64 countries that were ex-colonies and for which we have settler mor- tality, protection against expropriation risk, and GDP data (this is smaller than the sample in Figure 1). The GDP per capita in 1995 is PPP adjusted (a more detailed discussion of all data sources is provided in Appendix Table A1). Income (GDP) per capita will be our measure of economic outcome. There are large differences in income per capita in both the world sample and our basic sample, and the standard devia- tion of log income per capita in both cases is 1.1. In row 3, we also give output per worker in 1988 from Hall and Jones (1999) as an alterna- tive measure of income today. Hall and Jones (1999) prefer this measure since it explicitly refers to worker productivity. On the other hand, given the difficulty of measuring the for- mal labor force, it may be a more noisy measure of economic performance than income per capita.}

Table \ref{tab:descriptive-stats} provides descriptive statistics for the key variables of interest. Our two main measures of Covid-19-related outcomes is Covid-19-related deaths per million and GDP growth rates in 2020. Data on Covid-19 deaths were collected from Data Repository Center for Systems Science and Engineering (CSSE) at Johns Hopkins University. GDP growth rates were obtained from the \citet{imf}. A more detailed discussion of all data sources is provided in Appendix Table \ref{tab:data_sources}. 

The Covid-19 Data Repository Center for Systems Science and Engineering (CSSE) at Johns Hopkins University provides daily-updated data on confirmed cases and deaths for all countries, which are sourced from administrative sources \citep{covid-data}. We use this data to obtain the total number of confirmed deaths in 2020, which is in the first row of Table \ref{tab:descriptive-stats}.

The World Economic Outlook report by the \citet{imf} contains data on the annual GDP growth rates (including predictions) valued at purchasing power parity from 1980 to 2025. Since the official GDP growth rate for 2020 is not yet publicly available, we use the GDP growth rate from 2019 to 2020. It is described in the second row of Table \ref{tab:descriptive-stats}. 

{\color{blue}
We use a variety of variables to capture in- stitutional differences. Our main variable, re- ported in the second row, is an index of protection against expropriation. These data are from Political Risk Services (see, e.g., William D. Coplin et al., 1991), and were first used in the economics and political science literatures by Knack and Keefer (1995). Political Risk Ser- vices reports a value between 0 and 10 for each country and year, with 0 corresponding to the lowest protection against expropriation. We use the average value for each country between 1985 and 1995 (values are missing for many countries before 1985). This measure is appro- priate for our purposes since the focus here is on differences in institutions originating from dif- ferent types of states and state policies. We expect our notion of extractive state to corre- spond to a low value of this index, while the tradition of rule of law and well-enforced prop-erty rights should correspond to high values. The next row gives an alternative measure, con- straints on the executive in 1990, coded from the Polity III data set of Ted Robert Gurr and associates (an update of Gurr, 1997). Results using the constraints on the executive and other measures are reported in Acemoglu et al. (2000) and are not repeated here.

The next three rows give measures of early institutions from the same Gurr data set. The first is a measure of constraints on the executive in 1900 and the second is an index of democ- racy in 1900. This information is not available for countries that were still colonies in 1900, so we assign these countries the lowest possible score. In the following row, we report the mean and standard deviation of constraints on the executive in the first year of independence (i.e., the first year a country enters the Gurr data set) as an alternative measure of institutions. The second-to-last row gives the fraction of the pop- ulation of European descent in 1900, which is our measure of European settlement in the col- onies, constructed from McEvedy and Jones (1975) and Curtin et al. (1995). The final row gives the logarithm of the baseline settler mor- tality estimates; the raw data are in Appendix Table A2.

The remaining columns give descriptive sta- tistics for groups of countries at different quar- tiles of the settler mortality distribution. This is useful since settler mortality is our instrument for institutions (this variable is described in more detail in the next section).}

To measure democratic institutions, we collect data from three sources. These are described in the third to fifth rows. The first variable is the democracy index compiled by the \citet{eiu}. It is constructed from five interrelated categories: electoral processes and pluralism, functioning of government, political participation, political culture, and civil liberties.The second variable is the representative government index from the Global State of Democracy 2019 compiled by \citet{gsd}. It measures the extent that popular elections for legislative and executive offices are contested and inclusive. The third variable is the polity index from Polity IV 2018 by the \citet{polity}. It is constructed by subtracting the autocracy score, which measures institutionalized autocracy, from the democracy score, which measures institutionalized democracy.

The next five rows give descriptive statistics for 

The last give rows are 

\textcolor{red}{Note: Copy and pasted the entire Data section from Frey's paper. Parts that are not from Frey's paper are pasted from Acemoglu's paper. Copy and pasted parts are in blue. The edited version for this paper is in black.}

{\color{blue}
We build a dataset allowing us to trace the daily spread of Covid-19 cases, government’s re- sponse to the pandemic, and the movement of people across 111 countries over the entire lockdown period to date. Data on movement and travel were collected from Google’s Community Mobility Reports, and matched with information on policy restrictions, testing, and tracing from the Oxford Covid-19 Government Response Tracker (OxCGRT) (Hale et al., 2020). Table 1 provides some summary statistics for the variables of interest in our analysis.}

We build a dataset allowing us to compare total Covid-19 deaths and GDP growth in 2020 across 156 countries. Data on Covid-19 deaths were collected from Data Repository Center for Systems Science and Engineering (CSSE) at Johns Hopkins University and GDP growth rates were obtained from the \citet{imf}. Table \ref{tab:descriptive-stats} provides descriptive statistics for the key variables of interest. 

{\color{blue}
The Google Community Mobility Reports provide daily data on Google Maps users who have opted-in to the ”location history” in their Google accounts settings across 132 countries. The reports calculate changes in movement compared to a baseline, which is the median value for the corresponding day of the week during the period between the 3rd of January and the 6th of February 2020. The purpose of travel has been assigned to one of the following cate- gories: retail and recreation, groceries and pharmacies, parks, transit stations, workplaces, and residential.

OxCGRT is a novel dataset which is published by the Blavatnik School of Government at the University of Oxford. It contains various lockdown measures, such as school and work- place closings, travel restrictions, bans on public gatherings, and stay-at-home requirements, etc. These measures are complied into a stringency index, which is constantly updated to re- flect daily changes in policy. This allows us to analyse policy changes as well as geographic mobility patterns on a daily basis. Data on testing policy and contact tracing is also taken from OxCGRT.}

The Covid-19 Data Repository Center for Systems Science and Engineering (CSSE) at Johns Hopkins University provides daily-updated data on confirmed cases and deaths for all countries, which are sourced from administrative sources \citep{covid-data}. We use this data to obtain the total number of confirmed deaths in 2020, which is in the first row of Table \ref{tab:descriptive-stats}.

The World Economic Outlook report by the \citet{imf} contains data on the annual GDP growth rates (including predictions) valued at purchasing power parity from 1980 to 2025. Since the official GDP growth rate for 2020 is not yet publicly available, we use the GDP growth rate from 2019 to 2020. It is described in the second row of Table \ref{tab:descriptive-stats}. 

{\color{blue}
To measure democratic institutions, we collect data from two sources. Following BenY-ishay and Betancourt (2014), who argue that democracy constitutes both civil and political rights, we use the civil and political rights country score from Freedom in the World 2020, compiled by Freedom House. The second variable is a dummy variable equal to 1 if a country classified as authoritarian, taken from Dictatorship Countries Population 2020, compiled by the World Population Review.}

To measure democratic institutions, we collect data from three sources. These are described in the third to fifth rows. The first variable is the democracy index compiled by the \citet{eiu}. It is constructed from five interrelated categories: electoral processes and pluralism, functioning of government, political participation, political culture, and civil liberties.The second variable is the representative government index from the Global State of Democracy 2019 compiled by \citet{gsd}. It measures the extent that popular elections for legislative and executive offices are contested and inclusive. The third variable is the polity index from Polity IV 2018 by the \citet{polity}. It is constructed by subtracting the autocracy score, which measures institutionalized autocracy, from the democracy score, which measures institutionalized democracy.

The next five rows give descriptive statistics for the variables we use as controls: absolute latitude, mean temperature, mean precipitation, GDP per capita and population density. 

The last four rows describe the IVs. Log European settler mortality is sourced from \citet{ajr}. The European settler mortality rate $M_i$ is the annualized deaths per thousand mean strength of European soldiers, bishops, and sailors stationed in the colonies during the seventeenth and nineteenth centuries. The remaining three are sourced from \citet{hj}. The fraction speaking European is the fraction that speaks English, French, German, Portuguese or Spanish as a mother tongue. The Frankel-Romer trade share is the predicted share of trade in a country's economy, calculated based on the gravity model of trade. These IVs are discussed in more detail in Section \ref{sec:main-2sls}. 


\end{comment}


\subsection{Data}
We build a dataset allowing us to compare Covid-19-related outcomes across 156 different countries. Table \ref{tab:descriptive-stats} provides descriptive statistics for the key variables of interest. 

The first outcome variable is Covid-19-related deaths per million. Data are sourced from the Covid-19 Data Repository Center for Systems Science and Engineering (CSSE) at Johns Hopkins University \citep{covid-data}. The repository provides daily-updated data on confirmed cases and deaths for all countries, which are aggregated from mostly administrative sources. We use this dataset to obtain the total number of confirmed deaths between January 1st, 2020 and December 31st, 2020. 

The second outcome variable is the change in GDP between 2019 and 2020, which we source from the \citet{imf}. The World Economic Outlook report by the IMF provide data on the annual change in GDP (including predictions) for 195 countries and 32 regions from 1980 to 2025. Real GDP is valued at purchasing power parity. Since the finalized real GDP growth rates between 2019-20 have not yet been published as of January 2021, we use the predicted growth rates as of October 2020. 

To measure democratic institutions, we adopt three different indices (which are reported in the third, forth and fifth rows): the Democracy Index by the Economist Intelligence Unit, the Representative Government Index by International IDEA and the Polity Index by the Center for Systemic Peace. 

The Democracy Index by the \citet{eiu} provides a measure of the state of democracy in 165 countries. Based on the view that conventional measures of democracy (such as Freedom House's index) have not been comprehensive enough, the index is constructed from five interrelated categories: electoral processes and pluralism, functioning of government, political participation, political culture, and civil liberties. The index does not incorporate levels of economic and social wellbeing. The specific value we use is from the Democracy Index 2019 report, and it is on a scale of 0 to 10. 
    
The Representative Government Index by \citet{gsd} provides a measure of the extent to which popular elections for legislative and executive offices are contested and inclusive. It is one of the five main indices issued in International IDEA's Global State of Democracy Indices; the other indices are fundamental rights, checks on government, impartial administration and participatory engagement (there is no composite score). This index is constructed by first aggregating three subattributes - clean elections, free political parties, and elected government - into a contestation index using Bayesian factor analysis, which is then multiplied by the fourth subattribute, inclusive suffrage. All attributes, including the final representative government index, is on a scale from 0 to 1. In this paper, we use the Representative Government Index issued in 2019. 
    
The Polity Index by the \citet{polity} is constructed by subtracting the autocracy score, which measures institutionalized autocracy, from the democracy score, which measures institutionalized democracy. Both the democracy and autocracy indices are on an 0-10 scale and thus the composite polity score ranges from -10 to 10. In particular, we use the polity scores from the Polity IV 2018 data series as it is the latest available as of January 2021. 

The next five rows give the variables which we use as controls. Absolute latitude, mean temperature and mean precipitation are variables that we use to control for climate. We also use GDP per capita and population density to control for country characteristics that could be related to Covid-19-related outcomes. 

The last four rows are the IVs. European settler mortality is the annualized deaths per thousand mean strength of European soldiers, bishops, and sailors stationed in the colonies between the seventeenth and nineteenth centuries. It is used as an IV for institutions by Acemoglu et al. The remaining three rows are IVs for social infrastructure used by Hall and Jones. The fraction speaking English and the fraction speaking European are the percentages of each country's population that speak English or one of the five main Western European languages (English, French, German, Portuguese and Spanish) as a mother tongue. The Frankel-Romer trade share in the last row is constructed using the gravity model of international trade, which is based on the idea that trade between countries is proportional to size and inversely proportional to the geographic distance between them. 
More details on the data sources are in Table \ref{tab:data_sources}. 
    
\subsection{Democracy is associated with worse Covid-19-related outcomes}

Table \ref{tab:ols} reports ordinary-least-squares (OLS) regressions of Covid-19-related deaths per million (in Panel A) and GDP change 2019-20 (in Panel B) against the Democracy Index, which is normalized by its standard deviation. The linear regressions are of the equation: 
    
    \begin{equation}
    \label{eqn:ols}
        Y_i = \mu_i + \alpha Democracy_i + \beta X^{'}_i + \epsilon_i
    \end{equation}
    
\noindent where $Y_i$ is the outcome variable (Covid-19-related deaths per million or GDP change 2019-20) in country $i$, $\mu_i$ is the intercept, $Democracy_i$ is the normalized Democracy Index, $X^{'}_i$ is a vector of other covariates, and $\epsilon_i$ is a random error term. The coefficient of interest here is $\alpha$, which captures the effect of democracy on Covid-19-related outcomes. We allow different weightings in our OLS regression: no weighting, weighting by population and weighting by total GDP. 
    
Panel A's columns (1), (2) and (3) show that there is a strong correlation between the democracy measure and Covid-19-related deaths per million. Column (1) shows that, without any weighting of observations, an increase in one standard deviation in the democracy index corresponds to an increase in 165.6 deaths per million attributed to Covid-19. Using robust standard errors, this estimate is significant at the 1\% level. When we weight the observations by total population in column (2) and total GDP in column (3), the coefficient on the democracy measure is even larger, estimating an increase of 180.8 deaths and 228.8 deaths per million per standard deviation, respectively. 
  
To get a sense of the magnitude of the effect of democracy on performance in the pandemic, let us compare two countries, Costa Rica, which has approximately the 88th percentile of the normalized democratic index measure in this sample, 3.7 (democracy index = 81.3), and Venezuela, which has approximately the 14th percentile of the democratic measure, 1.43 (democracy index = 28.8). The estimate in Panel A's column (2) of 180.8, which weighs each country by population, indicates that there should be on average a difference of 330 deaths per million attributed to Covid-19 between the two countries. In reality, the gap is around 390, with Costa Rica experiencing 428 deaths per million and Venezuela experiencing 36 deaths per million. Since the population-weighted average of total deaths attributed to Covid-19 is 237 per million, if the effect of an increase of 330 deaths per million between Venezuela and Costa Rica as estimated in Table \ref{tab:ols} were causal, it would imply a fairly large effect. 
  
A strong correlation also exists between the democracy index and percentage GDP change between 2019 and 2020, as shown in Panel B's columns (1), (2) and (3). Although the result in column (1) (where observations are not weighted) is not statistically significant, population- or gdp-weighted OLS regressions provide statistically significant estimates that indicate that a standard deviation increase in the democracy measure is associated with a -3.1 change in GDP when weighted by population, and a -2.2 change when weighted by GDP. Considering that the population and gdp weighted averages in GDP change are -4.1 and -4.6, respectively, the association of one standard deviation increase in the democracy index with a 2-3\% decrease in GDP is significant. 
    
Yet, one could argue that there are other correlates causing the discrepancies in Covid-19-related outcomes, such as climate or other country characteristics such as wealth and population density. Indeed, experimental evidence shows that Covid-19 is sensitive to temprature, humidity and solar radiation \parencite{chin}; these affect the ability of the virus to persist on surfaces and in air, and might have impacts on transmissions. Moreover, a country's wealth could mean that countries are better prepared or the general population density could make Covid-19 transmission easier. To address such concern, we add several controls - namely, absolute latitude, mean temperature, mean precipitation, GDP per capita, and population density. 

However, even after adding the controls, the coefficients remain large and statistically significant, with a 90.6 increase in deaths per million per standard deviation without any weighting, and a 149.1 and 213.6 increase in deaths when weighted by population and total gdp, respectively. In the case of GDP change between 2019 and 2020, the addition of these controls barely changes the estimates, with the population- and gdp-weighted OLS coefficients remaining statistically significant at the 1\% level and the magnitude changing by only 0.1. 

Overall, the results in Table \ref{tab:ols} show a strong association between democracy and worse Covid-19-related outcomes. Higher democratic levels are associated with more deaths attributed to Covid-19 as well as greater negative shocks to GDP.

Nevertheless, there are two important reasons for not interpreting this relationship as causal. The first reason is that there are many omitted determinants of differences in Covid-19-related outcomes that will naturally be correlated with democracy. The second is the concern that the democracy variable is measured with error, which creates attenuation. These issues could be solved if we had IVs for political systems. Such an IV must be an important factor in accounting for the variation among political systems, but have no direct effect on performance in the Covid-19 pandemic. Our discussion in section \ref{sec:intro} suggests that settler mortality during the time of colonization is a plausible IV. 

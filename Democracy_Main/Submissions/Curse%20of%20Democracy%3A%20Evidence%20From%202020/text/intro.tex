GDP growth in the US is -3.5\% during 2020, while that in China is 2.3\%. The number of Covid-19-caused deaths per million is more than 300 times higher in the US than in China. Explaining such vast differences in economic and health performance is a pressing issue for today's world.

% GDP annual growth rate in 2020 in US source: https://www.bea.gov/news/2021/gross-domestic-product-fourth-quarter-and-year-2020-second-estimate

% GDP annual growth rate in 2020 in China source: https://www.reuters.com/article/china-economy-gdp-idUSL1N2JT039

% China: 3.4 Covid-19-caused deaths per million in 2020
% US: 1070 Covid-19-caused deaths per million in 2020

An obvious distinction between the US and China is in whether the political system is democratic or autocratic. 
Democracy is widely believed to promote economic prosperity and the safety of life, but whether democracy causes better outcomes is becoming increasingly debatable. In 2020 and 2021, the US along with other major democracies such as the UK and France face historic recessions and death tolls. The democratic countries stand in stark contrast to China and other autocratic countries, posing a natural question: 

\begin{quote}
``\textit{Are democracies hampered by inherent inefficiency and political division - or do their openness and diversity make for a more effective mobilization...?}" (\emph{The New York Times,} ``The Virus Comes for Democracy," April 2, 2020)\footnote{\url{https://www.nytimes.com/2020/04/02/opinion/coronavirus-democracy.html}} 
\end{quote}

Our goal is to answer this question. We construct a dataset that contains both historical and present-day information on the demographic, economic, health, and geographic characteristics of most of the world’s countries.
 We analyze the data with five different instrumental variables (IV) strategies. Our bottom line is that stronger democracies cause greater GDP declines and higher Covid-19 mortality during 2020. The result is robust to a variety of considerations: (a) how to measure the level of democracy in a country, (b) how to weight countries, (c) whether to control for country characteristics, and (d) the sample definition, especially whether to include extreme countries such as the US or China. The major channel for democracy's adverse effect appears to be weaker and narrower containment policies at the beginning of the pandemic, rather than the speed of policy implementation. 

We start by looking at the cross-country correlation between the outcomes and a widely-used index for democracy by Freedom House. The index assesses each country's degree of political freedom and civil liberties. As reported in Figure \ref{fig:ols}, a standard deviation increase in the democracy index corresponds to a 2.3 percentage-point decrease in GDP between 2019-20. A standard deviation democracy increase is also correlated with 264 more Covid-19-related deaths per million, which is nearly 95\% of the global average. To facilitate the interpretation of the finding, a standard deviation change in the democracy index is equivalent to the political-regime difference between Iraq and Indonesia or Indonesia and France. %Both estimates are also statistically significant at the 1\% level. 

Does this association of democracy with worse outcomes have any causal meaning? To identify democracy's causal effect, we adopt five of the most influential IVs for political and social institutions: 
\begin{itemize}
    \item Mortality of European colonial settlers \citep{acemogluColonialOriginsComparative2001}
    \item Population density in the 1500s \citep{acemogluReversalFortuneGeography2002}
    \item Availability of crops and minerals \citep{easterlyTropicsGermsCrops2003}
    \item Fraction of the population speaking English, fraction of the population speaking a Western European language, and the Frankel-Romer trade share, a measure of how easy it is for a country to engage in foreign trade \citep{hallWhyCountriesProduce1999}
    \item British, French, and German legal origin \citep{portaLawFinance1998}
\end{itemize}
These IVs help identify the effects of political institutions by tracing back their origins to geographical and historical determinants such as the feasibility and incentives of colonial powers to invest in institution-building, the origin of the legal institution, and natural resource endowments. Indeed, first-stage regressions show that several of these IVs are significant drivers of the cross-country variation in today's democracy levels. 

Two-stage least squares (2SLS) estimates show that democracy causes the worse outcomes. For example, with European settler mortality as an IV, a standard deviation increase in the democracy index causes a 3.1 percentage-point decrease in GDP and 441 more Covid-19-related deaths per million during 2020. The magnitude of these estimates is substantial, as the global average GDP growth rate in 2020 is -5.7 percentage points. The average Covid-19-related deaths per million is 285. The estimates are also statistically significant at the 1\% level using robust 2SLS standard errors (s.e.). Other IV strategies give similar robust estimates, ranging from -2.5 to -3.5 percentage points for GDP growth and 297 to 441 for Covid-19-related deaths per million. 
Once we account for this democracy effect, countries in Europe, North America, or South America no longer have worse outcomes. 

% -3.1, -2.7, -3.5, -2.5, (-0.2)
% 441, 417, (550), 297, (1035)

Our finding is robust to various alternative specifications. Controlling for latitude, temperature, precipitation, population density, median age, and diabetes does not change the results.\footnote{We also test whether our results are driven by industrial composition by including the share of the service sector as a control variable. Our results change little. Results are available upon request.} 
The baseline results change little even if we use alternative indices for democracy or weight countries differently. Moreover, the adverse effect of democracy is robust to excluding the US and China from the sample. The weakness of democracy is therefore a global phenomenon. 

As potential mechanisms that underlie this democracy effect, we explore the severity, coverage, and speed of governmental containment policies. We quantify initial responses' severities using the Oxford COVID-19 Government Response Tracker's Containment Health Index, which measures the severity of containment responses across domains such as school closings, stay-at-home restrictions, and travel restrictions. We represent coverage by the number of domains that containment policies cover. We finally measure speed by the number of days between the 10th confirmed Covid-19 case and the introduction of any containment measure. 2SLS estimates using IVs for political regimes suggest that a stronger democracy causes significantly weaker and narrower containment policies at the beginning of the pandemic. Meanwhile, we do not observe a significant causal effect of democracy on response speed, which suggests that severity and coverage are more critical mechanisms. \\

\textbf{Related Literature.} Our work is at the intersection of two strands of the literature: the relationship among democracy, economic growth, and public health, and the economics of pandemics. Any cause of macroeconomic growth and national public health is difficult to identify due to omitted variable biases, measurement errors, and limited data size \citep{klenow1997economic, helpman2009mystery}. Classic cross-sectional regression studies claim that democracy's cumulative effect on economic growth may be negligible \citep{barroDeterminantsEconomicGrowth1997,przeworskiPoliticalRegimesEconomic,przeworskiDemocracyDevelopmentPolitical2000}. With more quasi-experimental research designs, however, later studies show that democracies experience more stable, long-term growth than non-democracies \citep{acemogluDemocracyDoesCause2018, papaioannouDemocratisationGrowth2008, perssonDemocracyDevelopmentDevil2006, perssonGrowthEffectDemocracy2007, quinnDemocracyNationalEconomic2001, rodrikDemocraticTransitionsProduce2005}. Similar findings exist for democracy's positive effects on health \citep{besleyHealthDemocracy2006a, kudamatsuHasDemocratizationReduced2012}. More broadly defined Western social institutions are also shown to have positive effects on economic growth \citep{acemogluColonialOriginsComparative2001, acemogluReversalFortuneGeography2002, easterlyTropicsGermsCrops2003, hallWhyCountriesProduce1999}. We are not aware of a prior study that shows a substantially negative causal impact of democracy. 

% Democracy's positive effect may be because democracies are more responsive to the public's demands in education, health, income redistribution, and public goods \citep{ baumPoliticalEconomyGrowth2003, doucouliagosDemocracyEconomicGrowth2008,baumInvisibleHandDemocracy2001, tavaresHowDemocracyAffects2001}. 

Other studies inspect more closely the mechanisms behind democracy’s effects. Some studies use regional differences in democratic representation to find that higher representation leads to greater investments in education and public health \citep{baumPoliticalEconomyGrowth2003, doucouliagosDemocracyEconomicGrowth2008,baumInvisibleHandDemocracy2001, tavaresHowDemocracyAffects2001}. Studies such as \citet{besleyPoliticalInstitutionsPolicyChoices2003} and \citet{burgessValueDemocracyEvidence2015} focus on how different electoral processes within countries lead to different income redistributions and provisions of public goods.

We also contribute to the exploding literature on the economics of pandemics. Many researchers attempt to explain the cross-country heterogeneity in Covid-19-related outcomes. Studies show that obedience to travel restrictions or compliance with social distancing differ by culture, social capital, government communication, and political systems \citep{freyDemocracyCultureContagion2020, giuliano2020compliance, bilginDemocracyCOVID19Outcomes2021, bosancianu_dionne_hilbig_humphreys_kc_lieber_scacco_2020, Schmelze2016385118}. None of them finds any root cause of Covid-19-related outcomes. 
%carrieriHealthWealthTradeOffCovid192020, 

We integrate these two strands of the literature to find that democracy causes worse economic and public health outcomes during 2020. To our knowledge, this paper seems to be the only study that shows any substantially adverse effect of democracy on any important outcome. 

We organize this paper as follows. Section \ref{sec:data} describes our data and provides descriptive statistics.  Section \ref{sec:ols} analyzes the correlation between Covid-19-related outcomes and democracy. Section \ref{sec:causal} presents our 2SLS estimates of the causal effect of democracy. After Section \ref{sec:robust} discusses alternative specifications and explores the channels behind democracy's effect, Section \ref{sec:conclusion} concludes. 

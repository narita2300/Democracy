%\subsection{Data}
We are interested in how the policy performance of different countries in 2020 depends on their political regimes. 
We investigate the question by using the following five types of data. Table \ref{tab:descriptive-stats} provides descriptive statistics. For further details on data sources, refer to Appendix Table \ref{tab:sources}. 

\textbf{Economic and public health outcomes.} The first outcome we look at is the GDP growth rate in 2020 from the \emph{World Economic Outlook} by the \citet{internationalmonetaryfundWorldEconomicOutlook2020}. These are estimates, as the official GDP growth rates for 2020 are still unavailable. Row 1 in Table \ref{tab:descriptive-stats} shows that the average GDP growth rate estimate is -5.7\%, the worst since the World War II, for our sample of 175 countries. GDP growth rates also differ drastically across countries, with a standard deviation of 7.1\%. Figure \ref{fig:ols} visualizes these patterns. 

Our second outcome is the total number of deaths per million attributed to Covid-19 in 2020 from the Covid-19 Data Repository Center for Systems Science and Engineering (CSSE) at Johns Hopkins University \citep{dongInteractiveWebbasedDashboard2020}. This database collects data on confirmed cases and deaths from local administrative sources. The mean of Covid-19-related deaths per million is 285, while the median is 81. The number of Covid-19-related deaths is also dispersed, with a standard deviation of 376. We focus on these outcomes because they are not only important but also less susceptible to selective reporting than other outcomes, such as the number of COVID-19 cases \citep{fisher2020assessing}.


\textbf{Democracy indices.} Measuring the extent of democracy is tricky. Our baseline measure of the democracy level is from the \emph{Freedom in the World 2020} by \citet{freedomhouseFreedomWorld20202020}. Analysts award a country 0 (smallest degree of freedom) to 4 (largest degree of freedom) points for 10 political rights indicators and 15 civil liberties indicators. The index is the sum of these scores. It is widely used in the economics and political science literature as a measure for democracy \citep{barroDeterminantsDemocracy, freyDemocracyCultureContagion2020, spilimbergodemocracyForeignEducation,  stepan2003arab}. For robustness, we also use indices by the \citet{centerforsystemicpeacePolity5AnnualTime2018} and the \citet{DemocracyIndex2020}. As shown in Table \ref{tab:descriptive-stats}, these democracy indices capture our intuitive notion of democratic countries. According to the indices, the most democratic countries are Australia, Finland, and Norway while the least democratic countries are Bahrain, Congo, and Eritrea. 

% Our baseline measure of the democracy level is from the \emph{Polity Project} by the Center for Systemic Peace. The scores captures the democratic qualities of governing institutions on a 21-point scale ranging from -10 (hereditary monarchy) to +10 (consolidated democracy). It is widely used in the economics and political science literature as a measure for democracy \citep{knackDoesForeignAidPromoteDemocracy2004,stepan2003arab,spilimbergodemocracyForeignEducation}. For robustness, we also use the freedom index by \citet{freedomhouseFreedomWorld20202020} and the democracy index by the \citet{DemocracyIndex2020}. As shown in Table \ref{tab:descriptive-stats}, these democracy indices capture our intuitive notion of democratic countries. According to the indices, the most democratic countries are Australia, Finland, and Norway while the least democratic countries are Bahrain, Congo, and Eritrea. 

\textbf{Country characteristics.} To control for country characteristics and weight countries, we collect country-level data for GDP, population, absolute latitude, mean temperature, mean precipitation, population density, median age, and diabetes prevalence. %Descriptive statistics for these variables are in rows 6 to 13. 
We source data from the United Nations, the World Bank and the International Diabetes Federation.  % \citet{foodandagricultureassociationoftheunitednationsFAOGlobalStatistical2020}, \citet{unitednationsdepartmentofeconomicandsocialaffairspopulationdivisionWorldPopulationProspects2019}, \citet{theworldbankgroupClimateChangeKnowledge2020}, \citet{theworldbankgroupGDPCurrentUS2020}, and \citet{internationaldiabetesfederationIDFDiabetesAtlas2019}. 

\textbf{IVs.} To identify the causal effect of democracy, we use five of the most widely-used IVs for political institutions, as listed in the introduction and further discussed below. For each  IV, we obtain and extend data from the original authors.  %Further discussion of these IVs are in Section \ref{sec:causal}.  

\textbf{Policy responses.} To assess how democracy influences policy responses, we use panel data for 160 countries from the Oxford COVID-19 Government Response Tracker (OxCGRT) \citep{haleGlobalPanelDatabase2021}. OxCGRT's Containment Health Index summarizes the severity and scope of government measures by rating responses across various aspects of civilian life. The last three rows in Table \ref{tab:descriptive-stats} show that the Bahamas implement the strictest containment policies at the beginning of the pandemic, while countries such as Algeria are the most lenient. Moreover, the Solomon Islands are the speediest in their response, while Thailand is the slowest. 